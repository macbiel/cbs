\documentclass[12pt]{article}
\usepackage[a4paper, total={7in,9.6in}]{geometry}
\begin{document}


%Big header
\renewcommand{\H}[1]{\Large\textbf{#1}\\[2mm]\normalsize}
%Small header
\newcommand{\h}[1]{\large\textbf{#1}\\[2mm]\normalsize}
%Paragraph break
\newcommand{\n}{\\[6mm]}
%textbf alias
\renewcommand{\b}[1]{\textbf{#1}}
%textit alias
\renewcommand{\i}[1]{\textit{#1}}

\raggedright

\H{Classnote 3:\\Batch effects, technical variables, and unwanted variation}
Maciej Bielecki\\[12mm]


In molecular biology, \b{batch effects} refer to changes in experimental data
induced by seemingly irrelevant non-biological factors, such as variance in
personnel, conditions within the laboratory, or minute differences between the
used reagents or instruments.
\n

As batch effects lead to unwanted variation in the data, they can influence
what conclusions can be derived from it, and thus need to be dealt with 
appropriately. Further confounding the matter is the fact that the multiple 
\b{technical variables} behind batch effects are typically not recorded in a
verbose manner, instead being conflated into a single \b{surrogate variable},
usually just the processing date of the sample.
\n

Broadly speaking, there are three stages in the statistical
analysis of batch effects: exploratory analysis, downstream analysis,
and diagnostic analysis \cite{leek10}.
\n

During \b{exploratory analysis}, samples are clustered and labeled by both 
biological variables, which are the actual subject of the research, and
technical variables (often surrogates). Based on the PCA of the data,
principal components which correlate to the technical variables are identified.
\n

During \b{downstream analysis}, the data is adjusted for identified batch
effects, typically through clustering, linear regression or with ComBat
(explained later). In the event that the recorded technical variables are
insufficient in explaining all the artefacts present within the data, which is
particularly likely if said variables were surrogates, surrogate variable
analysis may be carried out to learn more about the artefacts.
\n

Lastly, \b{diagnostic analysis} refers to re-clustering the adjusted data and
confirming whether or not batch effects have been eliminated.
\n

The aforementioned \b{surrogate variable analysis} (SVA) obviates the need to
know in advance which technical variables cause batch effects, as it
automatically estimates the sources of batch effects based on the data 
\cite{leek07}. The most well-known method for carrying out SVA is the
\texttt{sva} Bioconductor package \cite{sva}.
\n

\b{ComBat} is an algorithm for adjusting data for batch effects, designed to be
robust for data with small batch sizes \cite{johnson07}, available as part of
the aforementioned \texttt{sva} Bioconductor package. This algorithm assumes
an location/scale model (L/S model) for the data, defined as:
\[ 
	Y_{ijg} = 
	\alpha_{g} + 
	X\beta_{g} + 
	\gamma_{ig} + 
	\delta_{ig}\epsilon_{ijg},
\]
where $Y_{ijg}$ is the data matrix, $\alpha_{g}$ is the overall gene expression,
$X\beta_{g}$ is a matrix of sample conditions, $\gamma_{ig}$ and $\delta_{ig}$
are the additive and multiplicative batch effects respectively, with 
$\delta_{ig}$ further modified by noise $\epsilon_{ijg}$. $i$, $j$ and $g$
represent specific batches, samples and genes, respectively.
\n

The algorithm first standardizes the data, such that genes have similar overall 
means and variances. The standardized data is calculated as:
\[
	Z_{ijg} = 
	\frac
		{
			Y_{ijg} - 
			\hat{\alpha}_{g} - 
			X\hat{\beta}_{g}
		}
		{
			\hat{\sigma}_{g}
		},
\]
where $Z_{ijg}$ is the data after standardization, $\hat{\alpha}_{g}$ and
$X\hat{\beta}_{g}$ are estimates of the corresponding model variables, and
$\hat{\sigma}_{g}$ is the estimated standard deviation.
\n

Upon standardization, the batch effects' parameters are estimated via conditional
posterior means, and the data is adjusted based on those estimates as:
\[ 
	Y^{*}_{ijg} =
	\frac
		{
			Y_{ijg} - 
			\hat{\alpha}_{g} - 
			X\hat{\beta}_{g} - 
			\hat{\gamma}_{ig}
		}
		{
			\hat{\delta}_{ig}
		} + 
	\hat{\alpha}_{g} + 
	X\hat{\beta}_{g},
\]
where $Y^{*}_{ijg}$ is data post-adjustment, while $\hat{\gamma}_{ig}$ and
$\hat{\delta}_{ig}$ are estimates of $\gamma_{ig}$ and $\delta_{ig}$.
\\[20mm]


\begin{thebibliography}{9}

\bibitem{leek10}
	Leek JT, Scharpf R, Bravo H \i{et al},
	Tackling the widespread and critical impact of batch effects in high-throughput data,
	\i{Nat Rev Genet}, 11, 733–739, 2010.
\bibitem{leek07}
	Leek JT, Storey JD,
	Capturing Heterogeneity in Gene Expression Studies by Surrogate Variable Analysis,
	\i{PLoS Genet}, 3(9), 1724-1735, 2007.
\bibitem{sva}
	Leek JT, Johnson WE, Parker HS, Fertig EJ, Jaffe AE, Zhang Y, Storey JD, Torres LC,
	sva: Surrogate Variable Analysis,
	2010.
\bibitem{johnson07}
	Johnson WE, Li C,
	Adjusting batch effects in microarray expression data using empirical Bayes methods,
	\i{Biostatistics}, 8(1), 118-127, 2007.

\end{thebibliography}

\end{document}
